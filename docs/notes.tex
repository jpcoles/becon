\documentclass{article}

\newcommand{\U}[1]{\ensuremath{\mathrm{\ #1}}}

\begin{document}

\section{Units}
Let $N$ be the grid size along one axis.
\begin{equation}
\begin{array}{ccl}
G &=& 1 \\
L &=& 1\U{Mpc} = 3.08568025\times 10^{22}\U{m} \\
T &=& 1/H_0 \approx 4.323391200\times 10^{17}\U{s} \\
\hbar/m &=& \frac{\Delta x^2}{N}\frac{2\pi}{T} \\
\Delta x \Delta k &=& 2\pi / N. 
\end{array}
\end{equation}

%$G = 1$, $\hbar = 1$, $H_0 = 1$\\
%$[T] = 13.7\U{Gyr}$, \\
%$[L] = 56.36041080\U{m}$\\
%$[M] = 6.967031011\times 10^{-21}\U{kg}$ \\
%$[\rho] = 3.8915740058804702 \times 10^{-26}\U{kg/m^3}$
%
%\bigskip\noindent
%Using Maple to find $[L]$ and $[M]$:
%\begin{verbatim}
%> Gsi  := 6.673e-11;
%> hbar := 1.05457148e-34;
%> yr   := 3600 * 24 * 365.25 * s;
%> tick := 13.7e9 * yr;
%> f1   := lap^3 / tick^2 / marb = Gsi * m^3 / s^2 / kg;
%> f2   := marb * lap^2 / tick = hbar * m^2 * kg / s;
%> solve({f1,f2});
%\end{verbatim}

\section{Initial Conditions}

GRAFIC provides a fixed grid of the density fluctuations $\delta(r)$ and the
velocity field $v(r)$. The phase $\phi$ of the wave function $\Psi = R e^{i\phi}$
is related to $v$ by
\begin{equation}
v = \frac{\hbar}{m}\phi_x
\end{equation}
%
To find the initial wave function $\Psi_0$ we integrate
\begin{equation}
\oint \vec v \cdot dr = \frac{\hbar}{m}\int d\phi = \frac{\hbar}{m}\phi
\end{equation}
%
Since the curl of $v$ vanishes we can chose any path for the integral.

\section{FFT Leap Frog}

Drift, gravity, kick. Rinse and repeat.

\begin{equation}
\begin{array}{rcl}
%
\tilde\Psi_n &=& FT(\Psi_{n-1}) \\
\tilde\Psi'_n &=& \tilde\Psi_n\exp(\frac{-i\hbar k^2}{2m}\frac{\Delta t}{2}) \\
\Psi'_n &=& IFT(\tilde\Psi'_n) \\
\end{array}
%
\begin{array}{rcl}
\rho_n &=& |\Psi'_n|^2 \\
\tilde\rho_n &=& FT(\rho_n) \\
\tilde\Phi_n &=& \tilde\rho_n / k^2 \\
\Phi_n &=& IFT( \tilde\Phi_n) \\
\end{array}
%
\begin{array}{rcl}
\Psi_n &=& \Psi'_n\exp(\frac{-i\Phi_n\Delta t}{\hbar}) \\
\end{array}
%
\end{equation}


\section{Euler Equations}

Let $\rho = \rho(x,t)$ and $R^2 \equiv \rho$.
%
\begin{equation}
\label{wave}
\Psi = R e^{i\phi}
\end{equation}
%
\begin{equation}
\label{Schroedinger}
i\Psi_t = -\frac{\hbar}{2m}\Psi_{xx}
\end{equation}
%
Take the time derivative of Equation \ref{wave}:
%
\begin{equation}
\label{S.dt}
i\Psi_t = i\left(R_t e^{i\phi} + iR\phi_t e^{i\phi}\right)
        = \left(i\frac{R_t}{R} - \phi_t\right)\Psi
\end{equation}
%
Now the first and second spacial derivaties:
%
\begin{equation}
\Psi_x = R_x e^{i\phi} + iR\phi_x e^{i\phi}
\end{equation}
%
\begin{equation}
\label{wave.dxdx}
\begin{array}{rcl}
\Psi_{xx} &=& R_{xx} e^{i\phi} + iR_x\phi_x e^{i\phi}
          + iR_x\phi_x e^{i\phi} + iR\phi_{xx} e^{i\phi}
          - R\phi_x^2 e^{i\phi} \\
          &=& \Psi\left(\frac{R_{xx}}{R} + 2i\frac{\phi_xR_x}{R} + i\phi_{xx} - \phi_x^2\right)
\end{array}
\end{equation}
%
Combine Equations \ref{Schroedinger}, \ref{S.dt}, \ref{wave.dxdx}
%
\begin{equation}
i\frac{R_t}{R} - \phi_t = -\frac{\hbar}{2m}\left(\frac{R_{xx}}{R} + 2i\frac{\phi_xR_x}{R} + i\phi_{xx} - \phi_x^2\right)
\end{equation}
%
Identify real and imaginary parts:
%
\begin{equation}
\label{real}
\frac{R_t}{R} = -\frac{\hbar}{2m}\left(2\frac{\phi_xR_x}{R} + \phi_{xx}\right)
\end{equation}
%
\begin{equation}
\label{imag}
\phi_t = \frac{\hbar}{2m}\left(\frac{R_{xx}}{R} - \phi_x^2\right)
\end{equation}
 
Let $v \equiv \frac{\hbar}{m}\phi_x$. We seek a continuity equation of the form
%
\begin{equation}
\rho_t + \nabla (\rho v) = 0
\end{equation}
%
Multiplying through by $R^2$ and writing in terms of $v$ Equation \ref{real} becomes
\begin{equation}
R R_t + \frac{R^2}{2}v_x + R R_x v = 0
\end{equation}

We now seek an equation for the conservation of momentum. Take the spacial derivative of Equation \ref{imag}:
%
\begin{equation}
\phi_{xt} = \frac{\hbar}{2m}\left(\left(\frac{R_{xx}}{R}\right)_x - 2\phi_x\phi_{xx}\right)
\end{equation}
%
Multiplying through by $\hbar/m$ and writing in terms of $v$ we have
%
\begin{equation}
v_t + vv_x = \left(\frac{\hbar}{m}\right)^2\left(\frac{R_{xx}}{2R}\right)_x
\end{equation}
%
For more spacial dimensions we can write
%
\begin{equation}
\frac{d\vec v}{dt} 
  = \frac{\partial \vec v}{\partial t} + \vec v \cdot \nabla \vec v
  = \left(\frac{\hbar}{m}\right)^2 \vec \nabla \left(\frac{\nabla^2 \sqrt{\rho}}{2\rho}\right)
\end{equation}


%\begin{equation}
%\end{equation}

\end{document}
